% Options for packages loaded elsewhere
\PassOptionsToPackage{unicode}{hyperref}
\PassOptionsToPackage{hyphens}{url}
%
\documentclass[
]{book}
\usepackage{amsmath,amssymb}
\usepackage{lmodern}
\usepackage{ifxetex,ifluatex}
\ifnum 0\ifxetex 1\fi\ifluatex 1\fi=0 % if pdftex
  \usepackage[T1]{fontenc}
  \usepackage[utf8]{inputenc}
  \usepackage{textcomp} % provide euro and other symbols
\else % if luatex or xetex
  \usepackage{unicode-math}
  \defaultfontfeatures{Scale=MatchLowercase}
  \defaultfontfeatures[\rmfamily]{Ligatures=TeX,Scale=1}
\fi
% Use upquote if available, for straight quotes in verbatim environments
\IfFileExists{upquote.sty}{\usepackage{upquote}}{}
\IfFileExists{microtype.sty}{% use microtype if available
  \usepackage[]{microtype}
  \UseMicrotypeSet[protrusion]{basicmath} % disable protrusion for tt fonts
}{}
\makeatletter
\@ifundefined{KOMAClassName}{% if non-KOMA class
  \IfFileExists{parskip.sty}{%
    \usepackage{parskip}
  }{% else
    \setlength{\parindent}{0pt}
    \setlength{\parskip}{6pt plus 2pt minus 1pt}}
}{% if KOMA class
  \KOMAoptions{parskip=half}}
\makeatother
\usepackage{xcolor}
\IfFileExists{xurl.sty}{\usepackage{xurl}}{} % add URL line breaks if available
\IfFileExists{bookmark.sty}{\usepackage{bookmark}}{\usepackage{hyperref}}
\hypersetup{
  pdftitle={Turkey Trip Notes},
  pdfauthor={Brad Gray},
  hidelinks,
  pdfcreator={LaTeX via pandoc}}
\urlstyle{same} % disable monospaced font for URLs
\usepackage{color}
\usepackage{fancyvrb}
\newcommand{\VerbBar}{|}
\newcommand{\VERB}{\Verb[commandchars=\\\{\}]}
\DefineVerbatimEnvironment{Highlighting}{Verbatim}{commandchars=\\\{\}}
% Add ',fontsize=\small' for more characters per line
\usepackage{framed}
\definecolor{shadecolor}{RGB}{248,248,248}
\newenvironment{Shaded}{\begin{snugshade}}{\end{snugshade}}
\newcommand{\AlertTok}[1]{\textcolor[rgb]{0.94,0.16,0.16}{#1}}
\newcommand{\AnnotationTok}[1]{\textcolor[rgb]{0.56,0.35,0.01}{\textbf{\textit{#1}}}}
\newcommand{\AttributeTok}[1]{\textcolor[rgb]{0.77,0.63,0.00}{#1}}
\newcommand{\BaseNTok}[1]{\textcolor[rgb]{0.00,0.00,0.81}{#1}}
\newcommand{\BuiltInTok}[1]{#1}
\newcommand{\CharTok}[1]{\textcolor[rgb]{0.31,0.60,0.02}{#1}}
\newcommand{\CommentTok}[1]{\textcolor[rgb]{0.56,0.35,0.01}{\textit{#1}}}
\newcommand{\CommentVarTok}[1]{\textcolor[rgb]{0.56,0.35,0.01}{\textbf{\textit{#1}}}}
\newcommand{\ConstantTok}[1]{\textcolor[rgb]{0.00,0.00,0.00}{#1}}
\newcommand{\ControlFlowTok}[1]{\textcolor[rgb]{0.13,0.29,0.53}{\textbf{#1}}}
\newcommand{\DataTypeTok}[1]{\textcolor[rgb]{0.13,0.29,0.53}{#1}}
\newcommand{\DecValTok}[1]{\textcolor[rgb]{0.00,0.00,0.81}{#1}}
\newcommand{\DocumentationTok}[1]{\textcolor[rgb]{0.56,0.35,0.01}{\textbf{\textit{#1}}}}
\newcommand{\ErrorTok}[1]{\textcolor[rgb]{0.64,0.00,0.00}{\textbf{#1}}}
\newcommand{\ExtensionTok}[1]{#1}
\newcommand{\FloatTok}[1]{\textcolor[rgb]{0.00,0.00,0.81}{#1}}
\newcommand{\FunctionTok}[1]{\textcolor[rgb]{0.00,0.00,0.00}{#1}}
\newcommand{\ImportTok}[1]{#1}
\newcommand{\InformationTok}[1]{\textcolor[rgb]{0.56,0.35,0.01}{\textbf{\textit{#1}}}}
\newcommand{\KeywordTok}[1]{\textcolor[rgb]{0.13,0.29,0.53}{\textbf{#1}}}
\newcommand{\NormalTok}[1]{#1}
\newcommand{\OperatorTok}[1]{\textcolor[rgb]{0.81,0.36,0.00}{\textbf{#1}}}
\newcommand{\OtherTok}[1]{\textcolor[rgb]{0.56,0.35,0.01}{#1}}
\newcommand{\PreprocessorTok}[1]{\textcolor[rgb]{0.56,0.35,0.01}{\textit{#1}}}
\newcommand{\RegionMarkerTok}[1]{#1}
\newcommand{\SpecialCharTok}[1]{\textcolor[rgb]{0.00,0.00,0.00}{#1}}
\newcommand{\SpecialStringTok}[1]{\textcolor[rgb]{0.31,0.60,0.02}{#1}}
\newcommand{\StringTok}[1]{\textcolor[rgb]{0.31,0.60,0.02}{#1}}
\newcommand{\VariableTok}[1]{\textcolor[rgb]{0.00,0.00,0.00}{#1}}
\newcommand{\VerbatimStringTok}[1]{\textcolor[rgb]{0.31,0.60,0.02}{#1}}
\newcommand{\WarningTok}[1]{\textcolor[rgb]{0.56,0.35,0.01}{\textbf{\textit{#1}}}}
\usepackage{longtable,booktabs,array}
\usepackage{calc} % for calculating minipage widths
% Correct order of tables after \paragraph or \subparagraph
\usepackage{etoolbox}
\makeatletter
\patchcmd\longtable{\par}{\if@noskipsec\mbox{}\fi\par}{}{}
\makeatother
% Allow footnotes in longtable head/foot
\IfFileExists{footnotehyper.sty}{\usepackage{footnotehyper}}{\usepackage{footnote}}
\makesavenoteenv{longtable}
\usepackage{graphicx}
\makeatletter
\def\maxwidth{\ifdim\Gin@nat@width>\linewidth\linewidth\else\Gin@nat@width\fi}
\def\maxheight{\ifdim\Gin@nat@height>\textheight\textheight\else\Gin@nat@height\fi}
\makeatother
% Scale images if necessary, so that they will not overflow the page
% margins by default, and it is still possible to overwrite the defaults
% using explicit options in \includegraphics[width, height, ...]{}
\setkeys{Gin}{width=\maxwidth,height=\maxheight,keepaspectratio}
% Set default figure placement to htbp
\makeatletter
\def\fps@figure{htbp}
\makeatother
\setlength{\emergencystretch}{3em} % prevent overfull lines
\providecommand{\tightlist}{%
  \setlength{\itemsep}{0pt}\setlength{\parskip}{0pt}}
\setcounter{secnumdepth}{5}
\usepackage{booktabs}
\ifluatex
  \usepackage{selnolig}  % disable illegal ligatures
\fi
\usepackage[]{natbib}
\bibliographystyle{apalike}

\title{Turkey Trip Notes}
\author{Brad Gray}
\date{2021-09-20}

\begin{document}
\maketitle

{
\setcounter{tocdepth}{1}
\tableofcontents
}
\hypertarget{prerequisites}{%
\chapter{Prerequisites}\label{prerequisites}}

This is a \emph{sample} book written in \textbf{Markdown}. You can use anything that Pandoc's Markdown supports, e.g., a math equation \(a^2 + b^2 = c^2\).

The \textbf{bookdown} package can be installed from CRAN or Github:

\begin{Shaded}
\begin{Highlighting}[]
\FunctionTok{install.packages}\NormalTok{(}\StringTok{"bookdown"}\NormalTok{)}
\CommentTok{\# or the development version}
\CommentTok{\# devtools::install\_github("rstudio/bookdown")}
\end{Highlighting}
\end{Shaded}

Remember each Rmd file contains one and only one chapter, and a chapter is defined by the first-level heading \texttt{\#}.

\hypertarget{intro}{%
\chapter{Introduction}\label{intro}}

\hypertarget{colossae}{%
\chapter{Colossae}\label{colossae}}

\hypertarget{geography}{%
\section{Geography}\label{geography}}

Mount Cadmus is source of water

\hypertarget{culture}{%
\section{Culture}\label{culture}}

Red-purple dye = Colossinus

\hypertarget{history}{%
\section{History}\label{history}}

Tell not yet excavated

480BC Xerxes (Persian) invaded

Leading city 5/6 Century BC

300BC Heropolis built

60 AD - Earthquake decimated

5/6 Century AC -\textgreater{} people moved out

\hypertarget{biblical-references}{%
\section{Biblical References}\label{biblical-references}}

Paul took road through interior to Ephesus on 3rd missionary journey
Paul never met Colossains
Letter written by Paul from Rome - Delivered by Tichichus

\hypertarget{ephesians-4-instructions-to-families}{%
\section{Ephesians 4: Instructions to Families}\label{ephesians-4-instructions-to-families}}

\hypertarget{domestic-relationships-in-roman-culture}{%
\subsection{Domestic Relationships in Roman culture}\label{domestic-relationships-in-roman-culture}}

Aristotle Politics: Household management in its perfect form: master + slave, husband + wife,

449BC - Law on 12 Tablets

Pater Familia - held power of life and death for household members

\hypertarget{ephesians-4}{%
\subsection{Ephesians 4}\label{ephesians-4}}

Paul didn't invent Roman household code. Instructions are given first to wives, then husbands (pater familias), then children. Addressing wives first was unheard of in Roman society.

\hypertarget{marriage-types}{%
\subsection{Marriage types}\label{marriage-types}}

\begin{itemize}
\tightlist
\item
  With the hand - Man brings a women into his home (and she is subject to him) - 50\%
\item
  without the hand - Wife is not under husband's authority
\end{itemize}

\hypertarget{instructions-to-husbands}{%
\subsection{Instructions to husbands}\label{instructions-to-husbands}}

Husbands - Agapao - love without expecting anything in return. This was revolutionary in Roman culture, which was highly transactional. Many marriages were politically expedient

Work for master - LORD: Kuios

\hypertarget{slavery}{%
\subsection{Slavery}\label{slavery}}

Paul couldn't abolish slavery, as the whole economy of the Roman Empire depended upon slavery.

\hypertarget{instructions-to-children}{%
\subsection{Instructions to children}\label{instructions-to-children}}

Children - when asked to obey - assumes agency

\hypertarget{aphrodisia}{%
\chapter{Aphrodisia}\label{aphrodisia}}

Friday Morning September 17th

City of Marble ( 1 mile from marble quarry)

Origins as small village until 2nd centry BC
Growth as artisians began to move from Pergamum
Incorporated as city in 1st century BC
44BC Julius Cesar assinated
- Brutus and Cassius were responsible
- Octavian and Mark Anthony and \_\_\_ form triumverate
- Civil war between Octavian and Mark Anthoby vs Brutus and Cassius
- Aphrodisia supported Octavian
- Labienus and his followers supported Brutus and Cassius
- Labienus pillories Aphrodisia for their support of Octavian
42BC Octavian and Mark Anthoy defeat Brutus and Cassius (who commit suicide)
39BC Mark Anthony grants free city status to Aphrosisia for their loyalty to him
30BC Mark Anthony commits suicide after defeat by Octavian (Caesar Augustus)
Inscription in wall from Julius Cesar: ``I love this city and take it as my own''

\hypertarget{origin-stories}{%
\section{Origin Stories}\label{origin-stories}}

Homer writes Iliad and Odyssey, which forms origin story for Greece
Viril writes Aenid, which connects Greek and ROman origin stories.

Stort starts when Troy is burning (same scene as end of Iliad).
Aphrodite (Venus) is father of Aenis

\hypertarget{emperors-connected-with-aphrodite-in-their-own-origin-story}{%
\subsection{Emperors connected with Aphrodite in their own origin story}\label{emperors-connected-with-aphrodite-in-their-own-origin-story}}

Omrit - 2 miles outside of Cesari Phillipi

\begin{itemize}
\tightlist
\item
  Temple to Cesar Augustus
\item
  Status of Venus present
  =\textgreater{} Emperor identified with Aphrodite
\end{itemize}

\hypertarget{temple-to-adphodite}{%
\section{Temple to Adphodite}\label{temple-to-adphodite}}

Tetrapylon - Gateway into sanctuary - built 200AD

Temenos - Courtyard
Proneos - Front part
Neos - Holy of Holies

\hypertarget{museum}{%
\chapter{Museum}\label{museum}}

Obelisk - 125 different names of those providing for those in need.
Three categories of names:

\begin{itemize}
\item
  Jews (n=61)
\item
  Proselytes (n=3) Gentiles who desire to convert to Judiasm. \href{https://en.wikipedia.org/wiki/Proselyte}{Wikipedia}

  \begin{itemize}
  \tightlist
  \item
    (Baptized)
  \item
    Circumcision
  \item
    Kosher laws
  \end{itemize}
\item
  God-Fearing (n=52)
\end{itemize}

Gentiles who want to worship Yaweh but don't want to adopt culture of Jews. \href{https://www.biblestudymagazine.com/bible-study-magazine-blog/2016/6/3/who-were-the-god-fearers}{Link}

Acts 13:16

\begin{quote}
So Paul stood up, and motioning with his hand said:
``Men of Israel and you who fear God, listen.
\end{quote}

Acts 17:4

\begin{quote}
And some of them were persuaded and joined Paul and Silas, as did a great many of the devout Greeks and not a few of the leading women.
\end{quote}

Acts 17:17

\begin{quote}
So he reasoned in the synagogue with the Jews and the devout persons, and in the marketplace every day with those who happened to be there.
\end{quote}

\hypertarget{sebastieon}{%
\section{Sebastieon}\label{sebastieon}}

\hypertarget{aphrodite}{%
\subsection{Aphrodite}\label{aphrodite}}

\begin{itemize}
\item
  Goddess of Love. Usually depicted as sensual
\item
  Mother Goddess. Statue found in Temple to Aphodite portrays her as mother godess (in full length dress)
\end{itemize}

Sebastos = venerable one = Greek version of Augustun

The Sebasteion, excavated in 1979-81, was a grandiose temple complex dedicated to Aphrodite and the Julio-Claudian emperors . Its construction stretched over two generations, from c.~AD 20 to 60, from the reign of Tiberius to that of Nero. The complex was paid for by two prominent Aphrodisian families. It consisted of a Corinthian temple and a narrow processional avenue (90 x 14 m) flanked by two portico-like buildings, each three-storeyed (12 m high), with superimposed Doric, Ionic, and Corinthian orders. These North and South Buildings, which defined the processional avenue, carried marble reliefs in their upper two storeys for their whole length. The reliefs were framed by the columnar architecture so that the two facades looked like closed picture-walls. Some 200 reliefs were required for the whole project, and more than 80 were recovered in the excavation. They featured Roman emperors, Greek myths, and a series of personified ethne or `nations' of Augustus' world empire, from the Ethiopians of eastern Africa to the Callaeci of western Spain\href{http://aphrodisias.classics.ox.ac.uk/sebasteion.html}{link}

Temple to Aphrodite and Julio-Claudian emporers (except Caligula)

Reliefs recounted the deeds of the emperor.

Poema = Work of poetra

Eph 2:8-10

\begin{quote}
For by grace you have been saved through faith. And this is not your own doing; it is the gift of God, not a result of works, so that no one may boast.For we are his workmanship, created in Christ Jesus for good works, which God prepared beforehand, that we should walk in them.
\end{quote}

We are God's Sebastien - recounting of his deeds

\begin{quote}
``Do you know what it means that you are God's workmanship? What is art? Art is beautiful, art is valuable, and art is an expression of the inner being of the maker, of the artist. Imagine what that means. You're beautiful, you're valuable, and you're an expression of the very inner being of the Artist, the divine Artist, God Himself. You see, when Jesus gave Himself on the Cross, He didn't say, ``I'm going to die just so you know I love you.'' He said, ``I'm going to die, I'm going to bleed, for your splendor. I'm going to re-create you into something beautiful. I will turn you into something splendid, magnificent. I'm the Artist; you're the art. I'm the Painter; you're the canvas. I'm the Sculptor; you're the marble. You don't look like much there in the quarry, but I can see. Oh, I can see!'' Jesus is an Artist!'' And you beloved are His crowning achievement, His masterpiece!\footnote{Tim Keller}
\end{quote}

How do our lives (and deeds) communicate a message about who God is?

I Peter - Living stones buiolt together

Your body is a temple

I Cor 3

I Cor 6

=\textgreater{} Allow God to shape your life

\hypertarget{kingdom-of-priests}{%
\section{Kingdom of Priests}\label{kingdom-of-priests}}

\hypertarget{philadelphia}{%
\chapter{Philadelphia}\label{philadelphia}}

\hypertarget{background}{%
\section{Background}\label{background}}

1 - ``Gateway to the East''
2 - Promoted Hellenistic values

\hypertarget{culture-and-industry}{%
\section{Culture and industry}\label{culture-and-industry}}

Famous for grape production due to volcanic soil. Catacecaumene = ``burnt land''

\hypertarget{religion}{%
\section{Religion}\label{religion}}

Patron saint - Dionysus

\begin{itemize}
\tightlist
\item
  God of wine
\item
  God of the arts esp theater
\item
  God of Spring Break
\end{itemize}

\hypertarget{history-1}{%
\section{History}\label{history-1}}

Founded by Attelis II (ruled 168-158 BC). Nicknamed Philadelphus because elder brother ruled Attelid kingdom (Eumenes II) before him from 197-159BC. In the 160s Attelis II made a trip to Rome, and wanted him to overthrow his brother. Eumenes found out, and named his brother

\hypertarget{antalya}{%
\chapter{Antalya}\label{antalya}}

\hypertarget{history-2}{%
\section{History}\label{history-2}}

\begin{itemize}
\tightlist
\item
  Founded by Attlus II
\item
  190BC Antiochus III defeated at battle of MAgnesia by Romans + Pergamenes.
  For events of Antiochus III See Daniel 11:10-19
\item
  188 BC Treaty of Apamea
\item
  133 BC - Attalus III willed lands to Rome, but Atalya remained free
\item
  64 BC - Rome annexed Atalya by Pompey
\item
  40-39 BC Parthians occupied Antioch on the Orantes
\end{itemize}

\hypertarget{saul}{%
\section{Saul}\label{saul}}

(Look at how people are introduced``)

\begin{itemize}
\tightlist
\item
  Acts 7:5-8 They laid their coats at the feet of Saul
\item
  Acts 8:1-3 Saul approved of their execution. He put beleivers in prison.
\item
  Acts 21:39 (arrested on temple mount
\item
  Ats 22:3
\end{itemize}

\hypertarget{tarsus}{%
\subsection{Tarsus}\label{tarsus}}

171BC citizens of Tarsus began receiving Roman citizenship. 64BC Capital of Cilicia. Site of largest philosophy school in the empire

\hypertarget{gamiel}{%
\subsection{Gamiel}\label{gamiel}}

Acts 22:3 Studied under Gamiel in Jerusalem (required \$)

Gamiel - First rabbi (my teacher) given title of rabboni (our teacher)

\begin{itemize}
\tightlist
\item
  Grandson of Hillel
\end{itemize}

Hillel

\begin{itemize}
\tightlist
\item
  Died when Jesus was 20
\item
  More Progressive/tolerant
\end{itemize}

Shamai

\begin{itemize}
\tightlist
\item
  More conservative
\end{itemize}

Jesus tended to favor teaching of Hillel (except about marriage)

Acts 5:30-40 Gamiel (honored by all): "If (the Way) is from God, you will not be able to stop it, because you cannot oppose God.

Acts 9:1 Saul - asked for letters for Damascus (150 miles away) to take them as prisoners

Acts 22:3

Acts 22:3-21

Acts 20:9-18

Galatians 1:3-7

\hypertarget{antalya-1}{%
\chapter{Antalya}\label{antalya-1}}

Acts 9:34 Damascus Saul,Saul

Ez 1:28 Rainow / Glory -\textgreater{} fell face down

Acts 9:15 GO -\textgreater{} proclaim my names to the Gentiles

Gal 1:15-16 God set me apart from my mother's womb

Is 49 Called me from my mother's womb -\textgreater{} light to the Gentiles

Jer 1:5

Paul arrives in town -\textgreater{} goes to synagogue first

Acts 9:9

Acts 9:4

Jonah - 3 days in the whale

Fighting against the goads

Gamiel -\textgreater{} if the Way is from God, you cannot resist it, because you would be resisting God

Playwright from the day (Euripidies): futility of resisting the gods- kick against the goads.

Gal 1: Gospel is not of human origin. As zealous as any

\hypertarget{zeal}{%
\section{Zeal}\label{zeal}}

Hebrew: Kana'
Phineas
Num 25
EliJah
I kings 19:10 I have zealously served the Lord
Zealots - Intertestimental period
- Zealous and stopping at nothing
- Maccabean revolt AD6

Ps 69:8-9

\hypertarget{paul-went-to-arabia}{%
\section{Paul went to Arabia}\label{paul-went-to-arabia}}

I Kings 19:15 Lord says to Elijah: return to wilderness

Gal 1:17 I did not go up to Jerusalem but went to Arabia

Gal 4:25 Hagar is Mt Sinai in Arabia

(Province of Arabia is Siani penninsula - NOT Saudi Arabia)

\hypertarget{wilderness}{%
\section{Wilderness}\label{wilderness}}

God meets people in the wilderness - MOses, Issrael - Elijah - Paul

Paul spends 3 years in the wilderness getting his theology right. He was zealous in persecuting the Way, but needed to re-orient.

Acsts 13:13 Pamphylia

Acts 14:25-26 Perga -\textgreater{} Atylia -\textgreater{} Antioch

\hypertarget{sardis}{%
\chapter{Sardis}\label{sardis}}

Saturday Sept 18th morning

\hypertarget{gospel-of-john}{%
\section{Gospel of John}\label{gospel-of-john}}

\begin{itemize}
\tightlist
\item
  Written to Asia
\end{itemize}

\hypertarget{vineyards}{%
\section{Vineyards}\label{vineyards}}

Jn

Ancient vineyards run along the ground. In contact with ground, grapes can become contaminated with fungus (grape smut?) which stunts their growth

Modern vineyards raise the vines off the ground on wires.

Vinedresser in ancient times if they found contaminated grapes, wound life up vines on a rock on on a forked stick in order to allow the sun to dry off the grapes and combat grape fungus.

Every branch that bears fruit the vinedresser prunes, that it will bear more fruit.

Pruning cuts off leaves

\begin{itemize}
\tightlist
\item
  Allows sun to get to other leaves
\item
  Permits directing nutrients
\end{itemize}

=\textgreater{} We need to trim the good things in our lives to make energy for the great things.

Requires being clear on who you are and what you are uniquely called (and gifted) to do

``Good to Great'' - Enemy of great is good

Brad Gray - As a candidate for a pastoral job - re-wrote the job description and handed it back go searach committee.

Look at how many times Jesus disppointed those around him (people he didn't heal). ``That is why I have been sent''

Imperative : \textbf{remain in me}

\hypertarget{what-are-the-fruit}{%
\section{What are the fruit}\label{what-are-the-fruit}}

\hypertarget{good-works}{%
\subsection{Good works}\label{good-works}}

You will know (false prophets) by their fruit - Can bramble bear good fruit?

\hypertarget{fruits-of-the-spirit}{%
\subsection{Fruits of the Spirit}\label{fruits-of-the-spirit}}

Character as evidenced by good deeds

\hypertarget{john-15-17}{%
\section{John 15-17}\label{john-15-17}}

Where do the events in these chapters take place?

John 14: Let us leave (Upper Room)

John 18 - Jesus left -\textgreater{} corssed the Kidron Valley
Perhaps they left the Eastern Gate after leaving the temple, which would lead to the garden (also referred to as Geshsemene).

\hypertarget{did-john-15-17-take-place-in-the-temple}{%
\subsection{Did John 15-17 take place in the Temple}\label{did-john-15-17-take-place-in-the-temple}}

Temple was left unlocked during passover (so that the Messiah wouldn't be locked out of the temple)

Jesus was the new Moses

\begin{itemize}
\tightlist
\item
\end{itemize}

\hypertarget{sardis-site}{%
\chapter{Sardis Site}\label{sardis-site}}

Saturday Sept 18th morning

Large site

\hypertarget{history-3}{%
\section{History}\label{history-3}}

Founded 1400BC - Center of Lydian Kingdom

Roman province of Asia established 129BC

First community to start dying wool, created dice,

\hypertarget{heraclips-dynasty.}{%
\subsection{Heraclips Dynasty.}\label{heraclips-dynasty.}}

Earliest rulers: Heraclips - ruled 505 years and 55 kings

Last Heraclips king: Canalves - had beautiful wife. Demanded that his chief bodyguard see her naked to know how beautiful she was. Canalves hid Guyges in his wife's bedroom. Wife discovers Guyges and proposes two options to address her dishonor: Guyges would kill himself, or would kill her husband (for arranging dishonor), and she would marry him. Guyges would become first of Mermnad dynasty.

\hypertarget{mermnad-dynasty}{%
\subsection{Mermnad Dynasty}\label{mermnad-dynasty}}

Heroditis - Mermnad rulers invented coinage

Mining gold in Paltolis river runs between Acropolis and Necropolis

\hypertarget{battle-with-cyrus}{%
\section{Battle with Cyrus}\label{battle-with-cyrus}}

Croesus - Mermnad king. Ten years into his rein, debates challenging Cyrus (leader of Persian empire).

Inquires of Oracle at Delphi. Was told: \emph{If you cross the River Hales, a great kingdom would be destroyed}

547BC Croesus crosses the Haley river at Cappodicia to fight Cyrus. Battle ends in stalemate. The end of the fighting season ends, Croesus dismissed the allied armies and heads back to Sardis. Cyrus pursues Croesus and asks the allied armies to join him, ,but they refuse.

October 546BC Cyrus besieged Sardis for one year.

Persian solder observed a Lydian soldier climb down the wall to retrieve a dropped helmet. This observation identified an ara of vulnerability in the wall. A small team was able to breach the wall at night, and open the city gates.

334BC - Sardis remained in Persian control

Conquered by Alexander the Great

220 BC - Celucid king in control

Achaeus usurped autority and rebelled against Antiochus III and pronounced himself king.

Antiochus

\hypertarget{thyatira}{%
\chapter{Thyatira}\label{thyatira}}

Smallest city, but longest letter

(Why site a city where there is no natural protection?)

3000 BC Lydian town - located near site of Lydian worship of sun god Tyrimnus.

Seleucus I formed a miliary colon with Macedonian soldiers in 300s BC

190 BC Seleucids under Antiochus III held the city.

188 BC Treaty of Apamea resulted in control by the Attalids
129BC incorporated into Roman province of Asia

\hypertarget{culture-1}{%
\section{Culture}\label{culture-1}}

Manufacturing city: Bronze working, baking, shoemaking

Acts 16:19 Paul by the river\ldots{} Woman Lydia from Thyatira, craftsman of purple cloth. (Location would have been Philllipi)

\hypertarget{deeds}{%
\section{Deeds}\label{deeds}}

Word used 5x in letter: ergon \(\epsilon \rho \omega \nu\)

James

Faith - pistos \$ \pi \iota \sigma \tau \omega \sigma \$
Workd - ergon \(\epsilon \rho \omega \nu\)

\hypertarget{food-sacrificed-to-idols}{%
\section{Food sacrificed to idols}\label{food-sacrificed-to-idols}}

Acts 15:29

\begin{quote}
that you abstain from what has been sacrificed to idols, and from blood, and from what has been strangled, and from sexual immorality. If you keep yourselves from these, you will do well. Farewell.''
\end{quote}

Acts 21:25

\begin{quote}
But as for the Gentiles who have believed, we have sent a letter with our judgment that they should abstain from what has been sacrificed to idols, and from blood, and from what has been strangled, and from sexual immorality.''
\end{quote}

\hypertarget{pergamum}{%
\chapter{Pergamum}\label{pergamum}}

\$ /pi /epsilon /rho /gamma /alpha /mu /omega /sigma \$ = Height / Elevation / Citadel

\hypertarget{geography-1}{%
\section{Geography}\label{geography-1}}

Not a port city

Located on a mountain - easily defendable. No steep cliff like Sardis

\hypertarget{history-4}{%
\section{History}\label{history-4}}

Settlement 8th Century BC

Controlled by Alexander the great.

After Alexander's death, empire split among Alexander's generals.

\begin{itemize}
\tightlist
\item
  Lysimachus takes control of western Asia
\item
  Celecus takes control of territories to the East
\item
  Celuces attacks Lysimicus in order to expand his territody
\item
  Philestaerus - steward of the city of Pergamum for Lysimicus: Switches sides to Celecus just prior to the battle for Pergamum.
\item
  Celecus wins and rewards philestra, who became wealthy
\end{itemize}

Philestaerus becomes first Attelid kings.

212 BC: Alliance of Rome and Attalus I attacks Antiochus III the Great and wins. Rome allows Attalus to keep control of Pergamum (as they were skeptical that they could control a large modern city.)

190 BC Battle of Magnasia

189 BC Treaty of Apamea

133 BC

129 BC Pergamum become capitol of province of Asia. 150 AD Ephesus is on the ascendency and overshadows Pergamum. Ephesus has a port, which Pergamum does not.

31 BC Octavian defeats Mark Anthony i Battle of \_\_\_.

29 BC Pergamum asks Octavian to allow it to be the first city to construct a temple for emperor worship. Octavian agrees, on the condition that temples are also constructed at Ephesus and Nicea to Roma and \_\_\_.

\hypertarget{ius-gladii}{%
\subsection{Ius Gladii}\label{ius-gladii}}

Only Roman governors and the emperor were given the right of the sword, meaning the power to execute capital punishment. Roman legionnaires carried a double-egded sword. Emporer carried a pear-shaped sword ( to emphasize their power of life and death). Usual mechanism of capital punishment was the gladiatorial ring.

\hypertarget{throne-of-satan}{%
\section{Throne of Satan}\label{throne-of-satan}}

Sites of pagan worship in Pergamum which might be candidates for ``Throne of Satan'':

\hypertarget{temple-of-zeus}{%
\subsection{Temple of Zeus}\label{temple-of-zeus}}

\hypertarget{temple-of-athena}{%
\subsection{Temple of Athena}\label{temple-of-athena}}

\hypertarget{trajan-temple}{%
\subsection{Trajan Temple}\label{trajan-temple}}

Built 98-117 AD for emperor worship.

Original temple built in 29 BC for emperor worship has not been found.

26 AD 2nd temple built for Smyrna - Neocordis for Tiberius. This means that for 50 years, Pergamum was leading site for emperor worship.

98 AD Trajan selects Pergamum as site for construction of temple for emperor worship. Supported by Arches (similar to Herod supporting Temple in Jerusalem)

Ephesians 1-3: Chritian identity. Chapters 4-6: Christian behavior. Fits pattern of identity -\textgreater{} Calling -\textgreater{} Conduct.

Our fight is not against flesh and blood.

Armour of God is defensive

Roman soldiers forbidden to protect hamstrings (so that they would not be comofrtable running away from the enemy).

Need a community to stand firm

\hypertarget{temple-to-dionysus}{%
\subsection{Temple to Dionysus}\label{temple-to-dionysus}}

Theater began in ancient Greece

Springtime festival: Dionysus was recurrected from the dead each spring. Dionysus was patron saint of theater and the arts.

\hypertarget{section}{%
\subsubsection{}\label{section}}

\hypertarget{scripture-memorization}{%
\chapter{Scripture memorization}\label{scripture-memorization}}

With memorization: scripture is always available and ``inside''

\begin{enumerate}
\def\labelenumi{\arabic{enumi})}
\item
  Find arc and movement of the story line. Read several times
\item
  Memorize images of lines of scripture on the bible page
\item
  Speak it 7 times outloud - ready to go on to next verse
\item
  Break up into chunks, then repeat chunks together
\item
  when tricky- draw a picture or acronym (especially for lists)
\item
  Motions with hands to remember what's next. Muscle memory for motions
\item
  Memory temple
\end{enumerate}

Beatitudes:

\begin{itemize}
\tightlist
\item
  Poor in spirit (dust on head)
\item
  Mourn - (tear)
\item
  Humble (no nose in the air)
\item
  Hungry (mouth)
\item
  Merciful (no grabbing the neck)
\item
  Peacemakers (hands)
\item
  Persecuted (sawn in half)
\end{itemize}

\begin{enumerate}
\def\labelenumi{\arabic{enumi})}
\setcounter{enumi}{7}
\item
  Memory is a muscle
\item
  Hebrew has a cadence when reciting. Look for rhythmn and beat.
\item
  No substitution for repetition
\item
  Scriptures tend to be visual (Psalm 1 - Walk -\textgreater{} sit -\textgreater{} stand)
\end{enumerate}

\hypertarget{communication-resources}{%
\chapter{Communication Resources}\label{communication-resources}}

Carmine Gallo

\begin{itemize}
\tightlist
\item
  Talk like TED
\item
  Storytellers Secret
\end{itemize}

\end{document}
